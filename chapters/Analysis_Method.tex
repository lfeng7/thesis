\section{Analysis Method for Template Based AFB Measurement}  % ==> 2 Days  No.2
% 1. Motivation and introduction
% 2. Analysis Scheme
% 3. Implementation of fitting using Theta
\subsection{Motivation}

The likelihood approach is based upon the observation that the angular distributions resulting from the s-channel dominated $q\bar q$ subprocess and from the t-channel dominated $gg$ subprocess are quite distinct.  Additionally, the gluon structure functions are ``softer'', more peaked a low x, than are the quark structure functions.  Therefore, highly boosted $t\bar t$ pairs, those produced at large $x_\mathrm{F}$ or rapidity, are more likely to be $q\bar q$-produced and the boost direction is most likely to be the direction of the incident quark.  To define the variables, we let $x_1$ and $x_2$ be the momentum fractions of the incident partons ordered so that the net boost is positive, $x_\mathrm{F}=x_1-x_2>0$.  The invariant mass of the $t\bar t$ pair, $M$, is then related to the momentum fractions, $M^2 = x_1x_2s$, where $s$ is the square of the pp center-of-mass energy. The differential cross section for $t\bar t$ production is composed of three parts,
\begin{align}
\frac{d^3\sigma}{dx_\mathrm{F}dMdc_*} =&\frac{2M}{s\sqrt{x_\mathrm{F}^2+4M^2/s}}\biggl\lbrace\frac{d\sigma}{dc_*}(q\bar q;M^2)\left[D_q(x_1)D_{\bar q}(x_2)+D_q(x_2)D_{\bar q}(x_1)\right] \nonumber \\ &+ \frac{d\sigma}{dc_*}(gg;M^2)D_g(x_1)D_g(x_2)\biggr\rbrace + \frac{d^3\sigma}{dx_\mathrm{F}dMdc_*}(\mathrm{background})
\label{eq:totxsdef}
\end{align}
where $x_{1,2}=\pm x_\mathrm{F}+ \sqrt{x_\mathrm{F}^2+4M^2/s}$, $c_*\equiv\cos{\theta^*}$ and $\theta^*$ is angle between the initial state quark direction and the top direction in the $t\bar t$ cm frame , and where the tree-level cross sections for $q\bar q,\ gg \to t\bar t$ are
\begin{equation}
\frac{d\sigma}{dc_*}(q\bar q;M^2) = \frac{\pi\alpha_s^2}{9M^2}\beta\left[1+\beta^2c_*^2+\left(1-\beta^2\right)\right]
\label{eq:qqzerodef}
\end{equation}
and
\begin{equation}
\frac{d\sigma}{dc_*}(gg;M^2) = \frac{\pi\alpha_s^2}{48M^2}\beta\left[\frac{16}{1-\beta^2c_*^2}-9\right]\left\lbrace\frac{1+\beta^2c_*^2}{2}+(1-\beta^2)-\frac{(1-\beta^2)^2}{1-\beta^2c_*^2}\right\rbrace
\label{eq:ggdef}
\end{equation}
and where the top quark velocity in the cm-frame is $\beta=\sqrt{1-4m_t^2/M^2}$.  The $gg$ subprocess produces a more forward-peaked cross section which provides the primary discriminant in the separation of the $gg$ and $qq$ subprocesses.

This study will consider events that can have extra jets which implies that the $t\bar t$ pairs can have non-zero transverse momenta .  This is accommodated in NLO descriptions by using the Collins-Soper (CS) definition \cite{Collins:1977iv} of the production angle and by allowing the cross section to develop a (CS frame dependent) term corresponding to longitudinal gluon polarization,
\begin{equation}
\frac{d\sigma}{dc_*}(q\bar q;M^2) = K\frac{\pi\alpha_s^2}{9M^2}\beta\left[1+\beta^2c_*^2+\left(1-\beta^2\right)+\alpha\left(1-\beta^2c_*^2\right)\right]
\label{eq:qqnlodef}
\end{equation}
where $K$ is a normalization parameter and the average longitudinal polarization $\alpha$ is determined from a fit to a sample of generated events.

An asymmetric $q\bar q$ subprocess could be caused by several kinds of new physics that interfere with or augment the tree-level process   \cite{Cao:2010zb, Gresham:2011pa}.  Most of these can be characterized in leading order by a small generalization of the tree-level cross section,
\begin{align}
\frac{d\sigma}{dc_*}(q\bar q;M^2) = R\frac{\pi\alpha_s^2}{9M^2}\beta\biggl\lbrace&1+\beta^2c_*^2+\left(1-\beta^2\right)+\alpha \left(1-\beta^2c_*^2\right) \nonumber \\
&+2\left[1+\frac{1}{3}\beta^2+(1-\beta^2)+\alpha\left(1-\frac{1}{3}\beta^2\right)\right]A_\mathrm{FB}^{(1)}c_*\biggr\rbrace
\label{eq:qqonedef}
\end{align}
Note that the asymmetry is characterized by the slope of the linear term in $c_*$ and is labelled with the superscript $(1)$.  Next-to-leading-order QCD corrections are expected  \cite{Kuhn:1998kw} to produce an asymmetry of approximately 8\%.  A comparison of the ratio of the $c_*$-odd and even terms for the full NLO calculation and for the simple linear model given in equation~\ref{eq:qqonedef} with $A_\mathrm{FB}^{(1)} = 0.08$ is shown in Fig.~\ref{fig:qcd_comp}.  The black curves show the NLO calculation for three different values of $M$.  The red curves show the linear model for the same values of $M$.  It is clear that the linear model is fairly accurate at lower masses and is still a reasonable approximation at larger masses.  A test of this hypothesis was performed by fitting the full NLO angular distribution generated by Powheg to the form given in equation~\ref{eq:qqonedef} and by comparing the resulting linearized asymmetry with the asymmetry determined from counting the forward and backward top events.  The results are listed in Table~\ref{tab:afb_counting_fitting} for the full sample and for the 4-jet and 5-jet subsamples.  Excellent agreement is observed.
\begin{figure}[hbt]
  \begin{center}
    \includegraphics[width=0.5\linewidth]{other/QCD_comp.pdf}
  \caption{\small The ratio of the $c_*$-odd and even terms for the full NLO calculation and for the simple linear model given in equation~\ref{eq:qqonedef} with $A_\mathrm{FB}^{(1)} = 0.08$.  The black curves show the NLO calculation for three different values of $M$: 400~GeV (solid), 600~GeV (dashes), and 1000~GeV (dots).  The black dash-dot curve corresponds to $b$ quarks and should be ignored.  The red curves show the linear model with $A_\mathrm{FB}^{(1)} = 0.08$ for the same masses.}
    \label{fig:qcd_comp}
  \end{center}
\end{figure}

\begin{table}[hbt]
\begin{center}
\caption{\small \label{tab:afb_counting_fitting} The $q\bar q\to t\bar t$ forward-backward asymmetry as determined from a sample of Powheg NLO generated events by counting and by fitting to the linearized function.}
\vspace{3pt}
\begin{tabular}{|l|cc|}\hline
 Sample      & $A_{FB}$ (counting) & $A_{FB}^{(1)}$ (fitting) \\ \hline
All events   & $+0.0356\pm0.0015$  & $+0.0352\pm0.0013$       \\ 
4 jets only  & $+0.0903\pm0.0018$  & $+0.0900\pm0.0016$       \\ 
5 jets only  & $-0.0698\pm0.0026$  & $-0.0720\pm0.0023$       \\ 
\hline
\end{tabular}
\end{center}
\end{table}


The distributions in ($M$, $c_*$, $x_F$) for the $gg$ and $q\bar q$ initial states can be visualized by considering a sample of $t\bar t(j)$ events generated with Powheg for $pp$ collisions at $\sqrt{s}=8$~TeV.  Because an extra jet is allowed, there is also a substantial contribution from the process $qg\to t\bar t q$ which is larger in magnitude than the $q\bar q$ subprocess.  The mass, $\cos\theta^*$, and $x_F$ distributions for the three subprocesses are shown in Fig.~\ref{fig:distributions}.  Note that the $gg$ and $qg$ distributions are quite similar.  Because the asymmetry for $qg$ events is expected to be smaller than for $q\bar q$ events events \cite{Kuhn:1998kw} (see also Table~\ref{tab:alpha_tune}), the $gg$ and $qg$ subprocesses are combined into a single distribution function for the purpose of this work.  The $q\bar q$ mass distribution is somewhat narrower than the others.  The $q\bar q$ angular distribution is much flatter than the others due to t-channel pole that dominates the $gg$ and $qg$ cross sections.  Of key importance, the $x_F$ distribution of the $q\bar q$ events has a longer tail that helps to discriminate them and to correctly identify the incident quark direction.  The result of taking the longitudinal direction of the $t\bar t$ pair in the lab frame as the quark direction is shown in Fig.~\ref{fig:distributions}(d).  Defining $N_C$ as the number of correct assignments and $N_I$ as the number of incorrect assignments, the dilution factor $D=(N_C-N_I)/(N_C+N_I)$ is plotted vs $x_F$.  Note that it becomes large in the $q\bar q$ enriched region at large $x_F$.
\begin{figure}[hbt]
  \begin{center}
    \includegraphics[width=\linewidth]{other/distributions_powheg.pdf}
  \caption{\small The mass (a), $\cos\theta^*$ (b), and $|x_F|$ (c) distributions for the subprocesses $gg/qg/q\bar q\to t\bar t(j)$.  The result of taking the longitudinal direction of the $t\bar t$ pair in the lab frame as the quark direction is shown in panel (d).  Defining $N_C$ as the number of correct assignments and $N_I$ as the number of incorrect assignments, the dilution factor $D=(N_C-N_I)/(N_C+N_I)$ is plotted vs $x_F$.  Note that it becomes large in the $q\bar q$ enriched region at large $|x_F|$.}
    \label{fig:distributions}
  \end{center}
\end{figure}

Because there can be ``feed-down'' from QCD processes that produce $t\bar t$ with more than one extra jet, we define the $gg$ label to include events produced from the $gg$, $qg$, $qq$, $\bar q \bar q$, and $q_i\bar q_j\ (\mathrm{flavor}\ i\neq \mathrm{flavor}\ j)$ subprocesses.

\clearpage
\subsection{Analysis Scheme}
\label{sec:analysis scheme}
It is possible to reconstruct the three key variables $x_\mathrm{r}$, $M_\mathrm{r}$, and $c_\mathrm{r}$ from lepton and 4(5)-jet final states.  The sign of the lepton tags the top vs antitop direction.  The direction of the pair along the beam axis can be taken as the likely quark direction for $q\bar q$.  Integrating over the pair pt (necessary only for the 5-jet cases), the data can be represented as a set of triplets in the reconstructed variables.  The distribution function of the reconstructed variables can be expressed as a convolution of the cross section defined in equation~\ref{eq:totxsdef} (with the $q\bar q$ cross section given by equation~\ref{eq:qqonedef}),

\begin{equation}
f(x_\mathrm{r},M_\mathrm{r},c_\mathrm{r}) = C \int dx_\mathrm{F}dMdc_* R(x_\mathrm{r},M_\mathrm{r},c_\mathrm{r}; x_\mathrm{F}, M, c_*)\varepsilon (x_\mathrm{F}, M, c_*) \frac{d^3\sigma}{dx_\mathrm{F}dM dc_*} 
\end{equation}

where $C$ is a normalization constant, $R$ is a ``resolution function'' that incorporates real detector resolution and parton shower effects, and $\varepsilon$ is an efficiency function.  The key point is that the linearity of the $c_*$-odd term in equation~\ref{eq:qqonedef} is not disturbed by the convolution and the linear coefficient $A_{FB}^{(1)}$ is unaffected.  The linearity of the problem also allows the fitting function to be represented by a set of nine {\bf parameter-independent} 3D histograms or templates.  These histograms can be constructed by appropriate weighting and re-weighting of a large sample fully digitized and reconstructed events from a simulation.  The $gg(qg)\to t\bar t(X)$ and background distributions $f_{gg}(x_\mathrm{r},M_\mathrm{r},c_\mathrm{r})$ and $f^j_\mathrm{bk}(x_\mathrm{r},M_\mathrm{r},c_\mathrm{r})$ can be extracted directly from fully simulated samples by binning in the reconstructed variables.  The various parts of the $q\bar q$ distribution can be constructed by re-weighting simulated data using generator-level variables to generate the weights and binning in reconstructed variables. 

To illustrate the re-weighting procedure, let's assume that we have a sample of fully simulated and reconstructed $q\bar q \to t\bar t$ events.  For simplicity, let's assume that $\xi,\delta=0$ in equation \ref{eq:qqonedef}.  If the simulation is tree-level, it generates the symmetric cross section \footnote{Due to the symmetrized weighting described below, NLO simulations generating asymmetric distributions can also be used.} given in equation~\ref{eq:qqnlodef} and we can create one 3D histogram or template simply by binning the events in the reconstructed variables.  We call this symmetric distribution $f_\mathrm{qs}(x_\mathrm{r}, M_\mathrm{r}, c_\mathrm{r}, Q)$ and normalize it by the total number of events.  We can generate the asymmetric distribution by applying the following weight to each simulated event using generator-level quantities,

\begin{equation}
w_\mathrm{a}(M^2, c_*) = 2\frac{1+\frac{1}{3}\beta^2+(1-\beta^2)+\alpha(1-\frac{1}{3}\beta^2)}{1+\beta^2c_*^2+\left(1-\beta^2\right)+\alpha\left(1-\beta^2c_*^2\right)}c_*
\end{equation}

and then binning the weighted events in the reconstructed quantities to produce the asymmetric distribution $f_\mathrm{qa}(x_\mathrm{r}, M_\mathrm{r}, c_\mathrm{r}, Q)$ with the same normalization as used for the symmetric distribution.  A simple three parameter likelihood fit to the real data would follow from the following four histograms,

\begin{align}
f(x_\mathrm{r},M_\mathrm{r},c_\mathrm{r}) =& \sum_jR^j_\mathrm{bk}f^j_\mathrm{bk}(x_\mathrm{r},M_\mathrm{r},c_\mathrm{r})+\biggl(1-\sum_jR^j_\mathrm{bk}\biggr )\biggl\lbrace \left(1-R_{q\bar q}\right) f_{gg}(x_\mathrm{r},M_\mathrm{r},c_\mathrm{r})\nonumber \\ &+R_{q\bar q}\left[f_\mathrm{qs}(x_\mathrm{r}, M_\mathrm{r}, c_\mathrm{r})+A_\mathrm{FB}^{(1)}f_\mathrm{qa}(x_\mathrm{r}, M_\mathrm{r}, c_\mathrm{r})\right]\biggr\rbrace
\label{eq:template_schemeone}
\end{align}

where the background fractions $R^j_\mathrm{bk}$, $q\bar q$ fraction $R_{q\bar q}$, and asymmetry $A_\mathrm{FB}^{(1)}$ are allowed to float.  Note that the backgrounds can be summed into a single distribution and represented by a single parameter or they can be subdivided into several parts represented by several fraction parameters.  This analysis should be done in bins or slices of $M_\mathrm{r}$ so that it is really a series of 3-parameter fits and extracts $A_\mathrm{FB}^{(1)}(M)$.  Due  to the limited statistics available in the 2012 data, mass binning of the parameters has not yet been implemented.  Note that this technique automatically accounts for resolution, dilution, migration, and acceptance effects so long as they are correctly modeled in the simulation.

\begin{figure}[hbt]
  \begin{center}
    \includegraphics[width=0.7\linewidth]{other/frames.pdf}
  \caption{\small The $t\bar t$ center-of-mass frame where system is presumed to be boosted in the direction of the proton with momentum vector $\vec p_1$ which determines the positive direction using the Collins-Soper definition of the production angle.}
    \label{fig:frames}
  \end{center}
\end{figure}

The acceptance for the moving $t\bar t$ pairs has a small subtlety that can be exploited to help distinguish the signal from the backgrounds.  The $t\bar t$ center-of-mass frame is shown in Fig.~\ref{fig:frames}.  The system is presumed to be boosted in the direction of the proton with momentum vector $\vec p_1$ and it determines the positive direction using the Collins-Soper definition of the production angle.  It is possible that the leptonically decaying $t$ or $\bar t$ is produced in the ``forward'' direction as shown on the left-hand side of the figure.  If the leptonic top decays to a positively (negatively) charged lepton, the sign of $c_*$ and $c_r$ are positive (negative).  Assuming that the detector locally accepts and reconstructs positive and negative charges with the same efficiency and resolution, the acceptance and resolution for the two cases are the same.  Similarly, the leptonically decaying $t$ or $\bar t$ can be produced in the ``backward'' direction as shown on the right-hand side of the figure.  Again, the sign of the lepton determines two cases that have the same efficiency and resolution.  However, the efficiency and resolution for the left and right cases are not in general the same.  A non-zero value of $A_\mathrm{FB}^{(1)}$ when combined with the acceptance difference would produce an asymmetry in the number of positively and negatively charged leptons observed in the sample.  The approach described above merges the two $c>0$ and the two $c<0$ cases to create truly symmetric and antisymmetric functions and cannot describe this effect.  It is, however, possible to split the problem by lepton charge instead.  This modifies equation~\ref{eq:template_schemeone} as follows,
\begin{align}
f(x_\mathrm{r},M_\mathrm{r},c_\mathrm{r},Q) =&  \sum_jR^j_\mathrm{bk}f^j_\mathrm{bk}(x_\mathrm{r},M_\mathrm{r},c_\mathrm{r})+\biggl(1-\sum_jR^j_\mathrm{bk}\biggr )\biggl\lbrace \left(1-R_{q\bar q}\right) f_{gg}(x_\mathrm{r},M_\mathrm{r},c_\mathrm{r},Q)\nonumber \\ &+R_{q\bar q}\left[f_\mathrm{qs}(x_\mathrm{r}, M_\mathrm{r}, c_\mathrm{r},Q)+A_\mathrm{FB}^{(1)}f_\mathrm{qa}(x_\mathrm{r}, M_\mathrm{r}, c_\mathrm{r},Q)\right]\biggr\rbrace
\label{eq:template_schemetwo}
\end{align}
where the functions are built using the lepton charge $Q$ information.  Because we desire to symmetrize and anti-symmetrize the $q\bar q$ fitting functions, the CP symmetries shown in Fig.~\ref{fig:frames} can be exploited to use each simulated event twice.  For each simulated event with lepton charge $Q$, generated angle $c_*$, and reconstructed angle $c_r$, the distribution functions for the coordinate $(x_\mathrm{r},M_\mathrm{r},c_\mathrm{r},Q)$ and $(x_\mathrm{r},M_\mathrm{r},-c_\mathrm{r},-Q)$ can be accumulated where the weights for the latter point assume a generated angle of $-c_*$.    The new distributions functions don't have definite symmetry until they are combined over lepton charge $Q$.  Due to the double-weighting, {\bf the charge-summed distribution functions have definite symmetry (or antisymmetry) even if the original unweighted simulation was not $\mathbf{c_*}$-symmetric.}  The function $f_{gg}$ describes the distribution of $gg$ and $qg$ events.  The $gg$ events are used symmetrically with 0.5 event accumulated in each of the $(c_\mathrm{r},Q)/(-c_\mathrm{r},-Q)$ bin pairs.  The $qg$ events are not symmetrized so that the final distribution function reflects their expected FB asymmetry.  The advantage of this formulation is that it can describe a charge asymmetry arising from the combination of a non-zero $A_\mathrm{FB}^{(1)}$ and an asymmetric acceptance.  More importantly, it accommodates the charge-asymmetric background which has significant contributions from $W$+jet events and single top events.  The accepted charge ratios of fully simulated and reconstructed semi-muonic top pair candidates from various signal and background processes are listed in Table~\ref{tab:njets}.  It is clear that including charge information increases the background discrimination power of the fitting procedure.

\begin{table}[hbt]
\begin{center}
\caption{\small \label{tab:njets} The sample fractions and lepton charge ratios for various signal and background processes from samples of fully simulated and reconstructed Powheg and MadGraph5 semi-muonic events.  The samples and selection criteria are described in Sections~\ref{sec:samples}-\ref{sec:selection}.}
\vspace{3pt}
\begin{tabular}{|lccc|}\hline
Process                                                     & Generator & Sample Fraction & $N(\mu^+)/N(\mu^-)$ \\ \hline
$q\bar q\to t\bar t(\mathrm{j})\to\mu+4(5)\mathrm{j}$       & Powheg    & 0.062           & 1.000$\pm$0.014                 \\ 
$gg(qg)\to t\bar t(\mathrm{j})\to\mu+4(5)\mathrm{j}$        & Powheg    & 0.731           & 0.998$\pm$0.004                 \\ 
$pp\to t\bar t(\mathrm{j})\to \mathrm{hadronic/dileptonic}$ & Powheg    & 0.106           & 1.018$\pm$0.011               \\
$W+\mathrm{jets}$                                           & Madgraph5 & 0.037           & 1.408$\pm$0.026                               \\ 
single top                                                  & Powheg    & 0.056           & 1.260$\pm$0.019                                         \\
$Z/\gamma+\mathrm{jets}$                                    & MadGraph5 & 0.009           & 1.045$\pm$0.039                       \\ \hline

\end{tabular}
\end{center}
\end{table}
