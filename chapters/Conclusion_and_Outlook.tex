\chapter*{Conclusion and Outlook}

In this thesis the Foward-backward asymmetry in $\ttbar$ production at Large Hadron Collider is measured using 19.7 $\fbinv$ of proton proton collision at 8 TeV. Only semileptonic channel of $\ttbar$ events are analyzed, with the final states of electron or muon , four or five jets with two originated from bottom quarks. 

A new template based measurement method is introduced in this thesis. Using this method, the parton level $\AFB$ originated from $\qqTT$ process is measured. In addition, the relative abundance of $\qqTT$ process among all $\ttbar$ production processes, denoted as $\rqq$ is measured simultaneously. 

The measurement is performed via a maximal likelihood fit that simultaneous fit both e+jets and $\mu$+jets events, with common parameters of interest, $\AFB$ and $\rqq$, and separate background process normalization estimation. 

Both $\AFB$ and $\rqq$ are found to be consistent with theoretical prediction given by standard model, within the uncertainty of the measurement. The dominant uncertainty in $\AFB$ measurement originates from the limited number of data events observed. The dominant uncertainty in $\rqq$ is of the systematical origin. 

One highlight of the measurement provide in this thesis is the direct measurement of $\AFB$ from $\qqTT$ process, allowing a closer comparison with the measurement of $\AFB$ in proton anti-proton collision in Tevatron.

To put the measurement result into perspective , we compared our result to other related measurements in LHC and Tevatron in recent years. As other measurements in LHC measured charge asymmetry, a different observable from what is measured in this thesis and in Tevatron measurements, instead of compare the results directly, we convert each individual measurement result to their relative difference from the expected value predicted by SM.

At the time of writing this thesis, the LHC Run2 has been extremely successful since its start in 2015. With a higher collision energy at 13 TeV indicating a larger cross section for $\ttbar$ production (832 pb at 13 TeV vs 245 pb at 8 TeV), and a much larger integrated luminosity recorded by CMS so far ( 95 $\fbinv$ compare with 19.7 $\fbinv$ ), about 14 times more $\ttbar$ events are expected using the LHC Run2 data collected so far. As the measurement presented in this thesis is limited by the size of data, a similar measurement is expect to have a 3 times smaller statistical uncertainty, which will greatly benefit the precise test of SM in the matter of $\AFB$

A challenge of measuring $\AFB$ in LHC Run2 is the increase of fraction of $\ttbar$ events originated from the symmetric gluon-gluon fusion process ( increase from 85\% to 90\% ) , which further dilute the expected charge asymmetry. Our approach, on the other hand , is less affected, as it managed to measure the $\AFB$ from $\qqTT$ process directly. 

In conclusion, the template based $\AFB$ measurement method proposed in this thesis is the first of its kind in CMS, and successfully measured the $\ttbar$ $\AFB$ using 8 TeV LHC data with competitive accuracy compare with previous measurements in CMS. It has even more potential in providing a more precise measurement using LHC Run2 data.