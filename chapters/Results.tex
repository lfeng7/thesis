%\chapter{Results} % ==> 3 Days
% 0. Overview / Pipeline of AN flow
% 1. Modelling of Signal and Background Events
% 2. Event Selection
% 3. Event reconstruction and template building revisit ( a quick summary ) 
% 4. Closure test
% 5. Systematics
% 6. Results

%uncertainty 
%results

\section{Sensitivity Studies}
\subsection{Gluon Polarization Study}
To tune the event weighting to use the Powheg and MagGraph samples listed in Table~\ref{tab:sim_samples}, generator-level $q\bar q\to t\bar t(j)$ events are fit to a distribution function derived from equation~\ref{eq:qqnlodef},
\begin{equation}
f_\mathrm{gen}(\alpha;M,c_*) = \frac{1+\beta^2c_*^2+\left(1-\beta^2\right)+\alpha\left(1-\beta^2c_*^2\right)}{2\left[2-\frac{2}{3}\beta^2+\alpha\left(1-\frac{1}{3}\beta^2\right)\right]}. \label{eq:wgt_test}
\end{equation}
to determine best values for $\alpha$.  The Powheg fit yields the surprising value $\alpha = -0.129\pm0.010$ indicating that Powheg generates $q\bar q\to t\bar t$ events with a steeper-than-tree-level angular distribution.  The presence of real longitudinal gluon polarization would manifest itself as a positive value for $\alpha$.  Note that the effect of positive or negative $\alpha$ is accounted in the definition of $A_{FB}^{(1)}$.  The goodness of fit can be demonstrated by applying the weight $f^{-1}_\mathrm{gen}$ to each event and plotting the resulting $|c_*|$ distributions for $\alpha = 0$ and $\alpha = -0.129$ as shown in Fig.~\ref{fig:unweight_test}(a-b).  The $\alpha=0$ ``unweighting'' shows a monotonic increase of about 8\% from smallest to large $|c_*|$ bin suggesting that the generated events are more strongly peaked at large $|c_*|$ than naive tree-level expectations.  Using $\alpha = -0.130$ removes the effect and leads to a maximum bin to bin variation of 1.7\%.  To test this further, the procedure is repeated by dividing the sample into 0-(extra)jet and 1-jet subsamples.  The effect of ``negative gluon polarization'' is seen more strongly in the 0-jet sample with a best fit of $\alpha=-0.256\pm0.011$ as shown in Fig.~\ref{fig:unweight_test}(c-d).  In the 1-jet sample, the presence of real longitudinal gluon polarization increases the best fit to $\alpha=0.143\pm0.019$ and is shown in Fig.~\ref{fig:unweight_test}(e-f).  The same procedure is performed on a sample of $q\bar q \to t\bar t(j,jj,jjj)$ events generated by MadGraph5.  A similar pattern is observed but the results, summarized in Table~\ref{tab:alpha_tune}, are not identical.  The Powheg and MadGraph5 predictions for the forward-backward asymmetry are also listed in Table~\ref{tab:alpha_tune}.  It is clear that the virtual NLO corrections contained in Powheg but not MadGraph5 are large and important.

\begin{figure}[hbt]
	\begin{center}
		\includegraphics[width=0.8\linewidth]{other/unweighting.pdf}
		\caption{\small The ``unweighted'' $|c_*|$ distributions (events weighted by $f_\mathrm{gen}^{-1}$) distributions of Powheg $q\bar q\to t\bar t(j)$ events for longitudinal gluon polarizations $\alpha=0$ [(a), (c), (e)] and best fit values [(b), (d), (f)].  The distributions are shown for samples containing: all events (a-b), 0 extra jets (c-d), and 1 extra jet (e-f).}
		\label{fig:unweight_test}
	\end{center}
\end{figure}

\begin{table}[hbt]
	\begin{center}
		\caption{\small \label{tab:alpha_tune} The best fit values for the longitudinal gluon polarization $\alpha$ for samples of Powheg(hvq) and MadGraph5 events.  The Powheg full NLO and MadGraph5 partial NLO expectations for the t-quark forward-backward asymmetry, the residual forward-backward asymmetry of the ``gluon-gluon'' sample from $q(\bar q)$-$g$ initial states, and the accepted $q\bar q$ event fractions are also listed.  Note that the ``$gg$'' asymmetries are smaller than the $q\bar q$ asymmetries by an order of magnitude.}
		\vspace{3pt}
		\begin{tabular}{|l|cccc|cccc|}\hline
			& \multicolumn{4}{c}{Powheg(hvq)} &  \multicolumn{4}{|c|}{MadGraph5} \\ 
			Sample    & $\alpha$      &  $A_{FB}$      & $A_{FB}^{gg}$ &  $R_{q\bar q}$ &  $\alpha$    &  $A_{FB}$      & $A_{FB}^{gg}$ &  $R_{q\bar q}$ \\ \hline
			All evts  & $-0.129(10)$  & $+0.0356(15)$  & $+0.0058(11)$ & 0.066          & $-0.173(7)$  & $-0.0283(27)$  & $-0.0026(11)$ & 0.093          \\ 
			\hline
		\end{tabular}
	\end{center}
\end{table}

We can also fit for $\alpha$ as a parameter that dependent on $t\bar t$ invariant mass \cite{AFB-13Tev-AN}. This allows for a more accurate description of the gluon longitudinal polarization based on the NLO MC simulation. We performed a binned likelihood fit of $c*$ distribution for simulated $q\bar q \rightarrow t \bar t$ events before any selection is applied. We divide simulated events by the range of $\beta$, and fit these events to get $\beta$ dependent $\alpha$ values. We then use the $\alpha$ acquired this way to make the asymmetric templates $f_{qa}$ as described in Eq.[\ref{eq:template_schemeone}]. 

The fit distribution and comparison to NLO MC simulated distributions are shown in Fig.[\ref{fig:alpha_fit}]. All the simulations are generated with proton-proton $\sqrt{s}=$ 13 TeV in the figures. 

\begin{figure}[hbt]
	\begin{center}
		\includegraphics[width=0.49\linewidth]{other/total_alpha_canv_amcnlo}
		\includegraphics[width=0.49\linewidth]{other/total_alpha_canv_powheg}
		\caption{\small Fit comparison of $c*$ for simulated $q\bar q \rightarrow t \bar t$ events generated by aMC@NLO (left) and Powheg (right) generators.The best fit values for $\alpha$ are shown in the legend. Simulation distribution is shown as cross, while best fit distribution is shown as solid lines. }
		\label{fig:alpha_fit}
	\end{center}
\end{figure}


\subsection{Closure Test}

The statistical power of the technique was investigated by simulating and fitting 2000 pseudo experiments of similar number of events in Data. We scan over a range of values of $A_{FB}$ and $R_{q \bar q}$, for every parameter value we generate 2000 pseudo experiments based on the statistical model described in Eq.\ref{eq:theta_exp_evts}, then fit the pseudo experiment with the same templates that generate pseudo-data. We than estimate the mean and spread of the fit results of all experiment by fitting with a Gaussian distribution. 

From the mean and standard deviation of fit value corresponding to every input value of parameters, we construct a Neyman band, which we use to extrapolate the confidence interval given the fit value of parameters from Data fit. We take the half of $68\%$ confidence interval as the statistical uncertainty of the template fit, which is indicated as dashed red lines in Fig.[\ref{fig:neyman}], The estimated statistical uncertainties are listed below:

\begin{itemize}
	
	\item $\sigma_{A_{FB}} = 0.50$
	\item $\sigma_{R_{q\bar q}} = 0.006$
\end{itemize}

An example distribution of pseudo experiments fit results for $A_{FB}$ and $R_{q\bar{q}}$ is shown in Fig.[\ref{fig:pseudo-ex}]. The Neyman construction is shown in Fig.[\ref{fig:neyman}]. From these plots we find the template fit has very small bias and the confidence interval extrapolated this way is close to the statistical uncertainty we get from THETA. 

\begin{figure}[hbt]
	\begin{center}
		\includegraphics[width=0.49\linewidth]{other/AFB_Neyman}
		\includegraphics[width=0.49\linewidth]{other/R_qq_Neyman}
		\caption{\small Neyman construction for $A_{FB}$(left) and $R_{q\bar q}$ (right). The dashed red line indicate the extrapolated fit result and the $\pm 1 \sigma$ value given the measured value. }
		\label{fig:neyman}
	\end{center}
\end{figure}

\begin{figure}[hbt]
	\begin{center}
		\includegraphics[width=0.49\linewidth]{other/pull_AFB_plus7pct}
		\includegraphics[width=0.49\linewidth]{other/pull_R_qq_minus7pct}
		\caption{\small Fit parameter distribution of 2000 pseudo experiments for $A_{FB}$(left) and $R_{q\bar q}$ (right).}
		\label{fig:pseudo-ex}
	\end{center}
\end{figure}

\clearpage

\section{Corrections and Systematic Uncertainties}
\label{sec:corrections}

The CMSSW simulation does not account for a number of known detector and experimental effects.  Standard CMS correction factors are applied to the simulated events to compensate for those deficiencies. In addition, there are uncertainties associated with theoretical models underling the event generation in both matrix element and parton showering stage. In this section we describe various corrections and associated systematic uncertainties related to our analysis.

\subsection{Experimental Uncertainties}
% Lumi, JES/JER, PU, Lepton Eff SF, B-tagging, QCD_norm, MC sys

\subsubsection{Jet Energy Scale}
Jet energy scales are a set of scale factors that correct the 4-momentum of jets reconstructed from CMS detector response to the particle level jet momentum. The corrections are applied sequentially in different stages which handles different aspects. The L1 Pile-up correction removes energy coming from pile-up events and is applied to both Data and MC. L2/L3 MC-truth correction correct the $p_T$ and $\eta$ of reconstructed jets to the particle level ones, applied to both Data and MC as well. Finally, L2/L3 residual corrections handles the difference in jets between MC and data.  

%https://twiki.cern.ch/twiki/bin/view/CMS/JECUncertaintySources#Main_uncertainties_2012_53X
We then estimate the systematic uncertainty by adjusting the jet energy scale factor depend on the $p_T$ and $\eta$ of jet. The amount of change in JES is according to the recommendation of JetMET PAG based on 20 $fb^{-1}$ of 2012 8 TeV Re-Reco Data \cite{JES_uncertainty} and listed in the following file.
\begin{itemize}
\item Winter14\_V5\_DATA\_Uncertainty\_AK5PFchs.txt
\end{itemize}

% new edit
The templates corresponding to $\pm 1\sigma$ from nominal value of JES for $gg\rightarrow t\bar{t}\rightarrow \mu+jets$ are shown below in Fig.\ref{fig:gg_JES_templates}. It can be seen that JES changes both the normalization and shape of this template. It turned out that JES is one of the dominate systematic uncertainties.

\begin{figure}[hbt]
  \begin{center}
    \includegraphics[width=0.49\linewidth]{mu/f_comb__gg__JES__cstar_sys}
    \includegraphics[width=0.49\linewidth]{mu/f_comb__gg__JES__mtt_sys}
    \includegraphics[width=0.49\linewidth]{mu/f_comb__gg__JES__xf_sys}

  \caption{\small The distribution of MC simulated $gg\/qg\rightarrow t\bar{t}\rightarrow \mu+jets$ events with $SF_{JES} = -1\sigma$ (blue), 0 (black) and $+1\sigma$ (red)}
    \label{fig:gg_JES_templates}
  \end{center}
\end{figure}


\subsubsection{Jet Energy Resolution}
% Official Recommendations:  https://twiki.cern.ch/twiki/bin/view/CMS/JetResolution
% code: https://github.com/lfeng7/diffmo/blob/semilep_AFB/Ntuplizer/plugins/jhu_hadHelper.h#L371
% code: https://github.com/lfeng7/diffmo/blob/semilep_AFB/Ntuplizer/plugins/smfaclookup.h
Measurements show that the jet energy resolution (JER) in data is worse than in the simulation and the jets in MC need to be smeared to describe the data. We use scaling method to correct the transverse momentum of a reconstructed jet, $p_T$, by a facor $w_{JER}$, defined below:
\begin{equation}
w_{JER}=1+(SF_{JER}-1)\frac{p_T-p_T ^{ptcl}}{p_T}
\end{equation}
where $p_T ^{ptcl}$ is the transverse momentum of jet clustered from generator-level particles, and $s_{\texttt{JER}}$ is the scale factor measured from data and MC comparison which is recommended by the JetMET POG\cite{JER_wiki} and listed Table.[\ref{tab:JER_SF}]

\begin{table}[htb]
\centering
\begin{tabular}{|cccc|}
\hline
$|\eta|$ range	 & down 		& central 	 &  up              \\ \hline
0.0-0.5     		 &  1.053       & 1.079   & 1.105             \\
0.5-1.1 	         &  1.071       & 1.099   & 1.127             \\ 
1.1-1.7          &  1.092       & 1.121   & 1.150              \\ 
1.7-2.3          &  1.162        & 1.208  & 1.254             \\
2.3-2.8          &  1.192       & 1.254   & 1.316              \\  \hline 
\end{tabular}
\caption{ Jet Energy Resolution scale factors and uncertainties for different $|\eta|$ range. }
\label{tab:JER_SF}
\end{table}

We evaluated the systematic by adjusting $SF_{JER}$ up and down as listed above to produce two more versions of templates for each MC sample. The effect of the JER and JES systematic on $t\bar{t}$ templates are also shown below.

% new edit
The templates corresponding to $\pm 1\sigma$ from nominal value of JES for $gg\rightarrow t\bar{t}\rightarrow \mu+jets$ are shown below in Fig.\ref{fig:gg_JER_templates}. It can be seen that JES changes the shape of this template, especially on $c^*$ and $M_{t\bar{t}}$ distributions.

\begin{figure}[hbt]
  \begin{center}
    \includegraphics[width=0.49\linewidth]{mu/f_comb__gg__JER__cstar_sys}
    \includegraphics[width=0.49\linewidth]{mu/f_comb__gg__JER__mtt_sys}
    \includegraphics[width=0.49\linewidth]{mu/f_comb__gg__JER__xf_sys}

  \caption{\small The distribution of MC simulated $gg/qg\rightarrow t\bar{t}\rightarrow \mu+jets$ events with $w_{JER}$= $-1\sigma$ (blue), 0 (black) and $+1\sigma$ (red)}
    \label{fig:gg_JER_templates}
  \end{center}
\end{figure}



\subsubsection{Pileup Reweighting}
% https://twiki.cern.ch/twiki/bin/view/CMS/PileupSystematicErrors

All simulated samples are reweighted to reflect the distribution of pileup events observed in data by applying a scale factor that depends upon the number of reconstructed pileup events. The scale factor is calculated for each bin by dividing the estimated number of true interactions in the 2012 dataset by the number of true interactions in the simulated samples. Pileup estimates for data are obtained from the pileup JSON file provided by the Physics Validation Team after taking into account the appropriate HLT path as described on the Pileup Reweighting TWiki \cite{pileup_reweighting_twiki}. The number of true interactions in simulation is shown on the left-hand side of Fig.~\ref{fig:MC_and_data_pileup} and the number of measured interactions in data is shown on the right-hand side, illustrating the discrepancy.

% new plots
\begin{figure}[hbt]
  \begin{center}
    \includegraphics[width=0.51\linewidth]{other/data_PU_plot}  
    \includegraphics[width=0.49\linewidth]{mu/PU_init_uncor_mu}
    \includegraphics[width=0.49\linewidth]{mu/PU_init_cor_mu}
  \caption{\small The distribution of simulated primary interactions in MC simulated $t\bar{t}\rightarrow \mu +jets$ events before applying PU reweighting (bottom left) and after reweighting (bottom right). The reference PU distribution from 2012 collision data in the top middle showing the discrepancy intended to be corrected for. }
    \label{fig:MC_and_data_pileup}
  \end{center}
\end{figure}

The effect of applying the reweighting brings the two measured pileup distributions much closer into agreement as illustrated in Fig.~\ref{fig:pileup_comparison}, which shows pileup in both simulation and data after reweighting where the signal and background simulations have all been scaled according to their luminosities and cross sections, and the total distribution normalized to the data.

% new plots
\begin{figure}[hbt]
  \begin{center}
    \includegraphics[width=0.49\linewidth]{mu/PU_uncorr_mu}
    \includegraphics[width=0.49\linewidth]{mu/PU_corr_mu}
    \includegraphics[width=0.49\linewidth]{el/PU_uncorr_el}
    \includegraphics[width=0.49\linewidth]{el/PU_corr_el}
  \caption{\small Measured pileup in simulation and data before reweighting (left) and after reweighting (right). The signal and background samples have been rescaled according to their luminosities and cross sections, and the entire distribution has been normalized to data. The simulated samples are pictured as stacked filled histograms, and the data are pictured as blue data points. The figures at top are from $\mu$+jets channel, and bottom are e+jets channel}
    \label{fig:pileup_comparison}
  \end{center}
\end{figure}

The systematic uncertainty associated with PU re-weighting mainly originate from the uncertainty of total cross-section of min-bias events as well as luminosity of bunch crossing. As recommended by the PVT POG, we apply 5\% uncertainty on the number of primary interactions of data to produce up and down weights for PU. Use the new weights we get the PU systematic templates for the fit. 

We have not include PU systematic yet and will added to the table soon.

\subsubsection{b-tagging Efficiency}
% https://twiki.cern.ch/twiki/bin/viewauth/CMS/BTagSFMethods#1a_Event_reweighting_using_scale
In our event selection a jet is tagged as a b jet if it passes a cut on its CSV discriminator value. However, the efficiency for a real b-jet to be tagged as a b quark is different in simulation and data, and so is the probability for a non-b jet to be misidentified as a b quark. A scale factor is applied to simulated events to correct for this discrepancy. The correction was done following the recommendation of BTEV POG \cite{Btag_POG}, using the method 1(a).

The scale factor (SF) is defined as the ratio of b-tagging efficiency for data and  MC. It is a function of jet flavor, $p_{T}$ and $\eta$. The b-tagging efficiency for a jet of flavor f and in the $(p_{T},\eta)$ bin of $(i,j)$ is defined as follows:
\begin{equation}
\varepsilon_{f} (i,j)=\frac{N_{f}^{b-tagged}(i,j)}{N_{f}^{total}(i,j)}
\end{equation}
Note here the b-tagging efficiency can be different for each MC sample. The weight that is applied for each event is then chosen as $w = \frac{P(data)}{P(MC)}$ where the probability of a given event in the MC distribution is
\begin{equation}
P(MC)=\prod_{i=tagged}\varepsilon _{i}\prod_{j=not\: tagged}(1-\varepsilon_{j})
\end{equation}
And the corrected probability for the distribution in data is
\begin{equation}
P(data)=\prod_{i=tagged}SF_{i}\varepsilon _{i}\prod_{j=not\: tagged}(1-SF_{j}\varepsilon_{j})
\end{equation}


\subsubsection{Muon Tracking Efficiency}

A tracking efficiency correction for muons is applied as an $\eta$-dependent scale factor. These scale factors are provided by the Tracking POG \cite{tracking_POG}.

\subsubsection{Muon Trigger, ID, and Isolation Efficiencies}

Muon trigger, ID, and isolation efficiences are corrected for by applying three scale factors, each dependent on the reconstructed number of primary vertices in the event as well as muon $\eta$ and $p_T$. These scale factors are provided by the Muon POG, and the procedure used is dicussed in detail on the twiki page for muon ID and isolation efficiencies \cite{muon_eff_twiki}.

\subsubsection{Electron ID Efficiency}
% https://twiki.cern.ch/twiki/bin/view/Main/EGammaScaleFactors2012#2012_8_TeV_Jan22_Re_recoed_data
We applied scale factors to correct the difference of electron cut-based ID efficiency between data and MC. The scale factors are recommended by EGamma POG\cite{Electron-ID-efficiency wiki}\cite{Electron-ID-efficiency AN} , which is measured from the following Data and MC samples using Tag-and-Probe Method:
\begin{itemize}
\item Data: DoubleElectron Run2012A+B
\item MC: DYJetsToLL-MadGraph (Summer12)
\end{itemize}
The SF measurement select events with opposite-sign di-electron events, with one electron as tag which pass tight electron cut-based ID and matched to the one leg of the trigger, and another electron as probe. The scale factors are measured in bins of $p_T$ and $|eta|$ and is applied event by event as a weight to correct MC to Data.

In systematic evaluations, we introduce a nuisance parameter with Gaussian prior distribution in the likelihood definition. The up and down templates are produced by applying the corresponding scale factors, instead of the central scale factors, for each event. 

\subsubsection{Electron Trigger Efficiency}
% https://twiki.cern.ch/twiki/bin/viewauth/CMS/KoPFAElectronTagAndProbe
We apply SF to correct the \texttt{HLT\_Ele27\_WP80} trigger efficiency of MC to Data. The scale factors are from sources recommended on CMS Top EGamma Coordination twiki page \cite{Top-Egamma-wiki}. It is measured by comparing MC simulation to 22Jan2013 ReReco Data, using tag and probe method\cite{Electron-Trigger-Efficiency wiki}\cite{Electron-Trigger-Efficiency AN}. The samples for the SF measurement is listed below:
\begin{itemize}
\item Data : \texttt{/SingleElectron/Run2012*-22Jan2013/AOD}
\item MC : \texttt{/DYJetsToLL\_M-50\_TuneZ2Star\_8TeV-madgraph-tarball \\
           /Summer12\_DR53X-PU\_S10\_START53\_V7A-v1/AODSIM}
\end{itemize}

Similarly to Electron ID efficiency SF discussed above, we applied trigger efficiency correction to MC and introduce a nuisance parameter in systematic evaluations. 


\subsubsection{QCD Modeling and Background Composition}

Due to the high cross section and wide variety of event types resulting from multijet QCD processes, Monte Carlo simulations cannot be generated with sufficient luminosity to provide a reasonable approximation of this background shape. Therefore a data-driven method has been implemented to estimate the shape of the QCD background.

The nature of the method is to build template distributions from each of the existing simulated samples (both signal and background) in a sideband of the lepton isolation variable. The sideband used for muons is $0.13<\mathrm{PF}_\mathrm{iso}/p_{T}<0.20$ and the sideband used for electrons is defined as  $0.2<\mathrm{PF}_\mathrm{iso}/p_{T}<1.2$. These sideband regions are inversions of the lepton selection cuts and are designed to provide a sample enriched with multijet QCD events.

In the muon+jets channel with Run2012A-D data, it was found that only 485 events were selected in the lepton isolation sideband. Additionally, all of these events could be accounted for by the events selected from the existing signal and background simulations, indicating that QCD multijet contribution to the background is negligible in this channel. So for muon+jets channel we do not consider QCD multijets background.

On the other hand, QCD multijets process is not negligible for $e+jets$ channel. So we estimate the distribution of QCD process by using the distribution of observed data events in the lepton selection sideband region. This is assuming the distribution of QCD events is similar regardless of if the fake electron is isolated or non-isolated. In addition, we estimate the event rate for QCD process in signal region by using ABCD method, including the uncertainty of this estimation. We then introduce a nuisance parameter $R_{QCD}$ with a log-normal prior, so the data-driven QCD background can be scaled. The width of the prior distribution of $R_{QCD}$ is chosen as the percentage uncertainty estimated from ABCD method. 

The QCD multijet background process modeling is described in more details in Appendix.[\ref{sec:data driven qcd}]

\subsection{Theoretical Uncertainties}
% PDF, top pT reweighting

\subsubsection{Top $p_{T}$ Reweighting}

The normalized differential top-quark-pair cross section analysis in the CMS Top Group found a persistent inconsistency between the shapes of the individual top-quark $p_{T}$ distributions in simulation and data, while the NNLO approximated calculation \cite{Kidonakis2012} provides a reasonable description \cite{pT_reweighting_TWiki}. Therefore an individual top-quark $p_{T}$ dependent event scale factor has been derived to correct this shape. The scale factors recommended for use with 8TeV data for lepton plus jets events are 
\begin{equation}
\mathrm{weight} = \sqrt{e^{0.318-0.00141(p_{T_{t}}+p_{T_{\bar{t}}})}}
\end{equation}

Here $p_{T_t}$ and $p_{T_{\bar t}}$ are the generator-level transverse momenta of the individual top- and antitop-quarks respectively. Note that the application of this event scale factor does not conserve the $t\bar{t}$ cross section and this change in total cross section must be removed when renormalizing the $t\bar{t}$ samples by luminosity and cross section to derive expectations of $R_\mathrm{bk}$ and $R_{q\bar q}$. More details about the reweighting procedure, its motivation and investigation, and its application to samples of 7 TeV data can be found on the Top Quark Group's TWiki page \cite{pT_reweighting_TWiki}. The value of this event scale factor for semileptonic events is pictured in Fig.~\ref{fig:top_pT_SF} as a function of $c_\mathrm{r}$, $x_\mathrm{r}$, and $M_\mathrm{r}$. It shows that top $p_T$ reweighting is correlated with $c_r$ and $M_{t\bar t}$ for $t\bar{t}$ events. As a result, applying the reweighting changes the distribution of $t\bar{t}$ events.

\begin{figure}[hbt]
  \begin{center}
    \includegraphics[width=0.48\linewidth]{mu/xf_SF_mu}
    \includegraphics[width=0.48\linewidth]{el/xf_SF_el}   
    \includegraphics[width=0.48\linewidth]{mu/Mtt_SF_mu}
    \includegraphics[width=0.48\linewidth]{el/Mtt_SF_el}
    \includegraphics[width=0.48\linewidth]{mu/cstar_SF_mu}
    \includegraphics[width=0.48\linewidth]{el/cstar_SF_el}
  \caption{\small The top $p_{t}$ reweighting event scale factor as a function of $c_\mathrm{r}$ (top), $x_\mathrm{r}$ (middle), and $M_\mathrm{r}$ (bottom) for a sample of aMC@NLO simulated semileptonic $t\bar{t}$ events.  The figures at the left are from $\mu$+jets channel, at right are e+jets channel.}
    \label{fig:top_pT_SF}
  \end{center}
\end{figure}

In subsequent template fit we apply top $p_T$ re-weighting as a default, for our nominal fit. We estimate the systematic uncertainty associated with top $p_T$ re-weighting by measuring the difference of central fit value of $A_{FB}$ and $R_{q\bar{q}}$ with or without applying the top $p_T$ weights. We found that top $p_T$ reweighting is a dominate systematic in this measurement. 

\subsubsection{MC systematics}

The effect of having limited sized MC templates on the fit is discussed here. This systematic uncertainty has not been estimated yet and will be added in next version of the notes.

\subsubsection{Parton Distribution Functions}

We estimate the systematic uncertainty from PDF of protons by producing up/down templates based on all alternative PDF sets for each MC sample. For example, for our signal $t\bar{t}$ sample which is produced with \texttt{aMC@NLO+CTEQ66} , for every event we take the PDF weights that are maximally below ( $w_-$ ) and above ($w_+$) the value of nominal weight ($w_0$) to produce $w_{down}=\frac{w_-}{w_0}$ and $w_{up}=\frac{w_+}{w_0}$ . Then we re-weight the nominal templates using these two set of weights to produce systematic templates for PDF uncertainty. 

\subsection{Evaluation method and uncertainty table}

Once we have systematic templates corresponding to each of the uncertainty sources, we propagate the uncertainties to the measured parameters by taking the following approach. 

As mentioned in Section.\ref{sec:fitter} for every systematic uncertainty sources we introduce a nuisance parameter with Gauss prior. The expected distribution can be interpolated from up,down and nominal templates provided. We first perform the template fit by fixing all nuisance parameters corresponding to systematics to the nominal value, only allowing other parameters to float. Then we allow each systematic nuisance parameter to float at a time, and take the difference between the new measured parameter value and nominal value as the systematic uncertainty from the respective source. Finally we add all systematic uncertainties in quadrature  as the total systematic uncertainties. 

The complete table of systematic uncertainties for both parameter of interest and other important nuisance parameters are listed below, in Table.[\ref{tab:sys-full}],[\ref{tab:sys-err}]

\begin{table}[htb]
\centering
\begin{tabular}{c|cc|cccc}
Systematics &      $A_{FB}$ &   $R_{q\bar{q}}$ & $R_{other\_bkg\_\mu}$ & $R_{other\_bkg\_el}$ & $R_{WJets\_\mu}$ & $R_{WJets\_el}$ \\
\hline
Nominal         &  0.045 &   0.12 &          0.089 &          0.101 &      0.007 &      0.011 \\
\hline
B-Tagging Eff.  &  0.044 &  0.119 &          0.089 &          0.101 &      0.007 &      0.011 \\
Lepton ID Eff.  &  0.045 &   0.12 &          0.088 &          0.101 &      0.007 &      0.011 \\
Lepton Iso Eff. &  0.045 &   0.12 &          0.089 &          0.101 &      0.007 &      0.011 \\
Tracking Eff.   &  0.045 &   0.12 &          0.089 &          0.101 &      0.007 &      0.011 \\
Trigger Eff.    &  0.045 &   0.12 &          0.089 &          0.101 &      0.007 &      0.011 \\
\hline
JES             &   0.05 &  0.114 &          0.106 &           0.12 &      0.011 &      0.014 \\
JER             &  0.039 &  0.118 &          0.095 &          0.104 &      0.008 &      0.011 \\
PDF             &  0.036 &   0.12 &          0.095 &          0.104 &      0.008 &      0.011 \\
\hline
top $p\_T$         &  0.034 &  0.108 &          0.073 &          0.084 &      0.006 &       0.01 \\
\hline
\end{tabular}
\caption{Central value of all fit parameters with each type of systematic nuisance parameters turn on at a time.}
\label{tab:sys-full}
\end{table}

\begin{table}[htb]
\centering
\begin{tabular}{c|cc|cccc}
Systematics &    $\sigma_{AFB}^{sys}$ & $\sigma_{R_{q\bar{q}}}^{sys}$ & $\sigma_{R_{other\_bkg\_\mu}}^{sys}$ & $\sigma_{R_{other\_bkg\_el}}^{sys}$ & $\sigma_{R_{WJets\_\mu}}^{sys}$ & $\sigma_{R_{WJets\_el}}^{sys}$  \\
\hline
B-Tagging Eff.  &   0.001 &    0.001 &                  0 &                  0 &              0 &              0 \\
Lepton ID Eff.  &       0 &        0 &              0.001 &                  0 &              0 &              0 \\
Lepton Iso Eff. &       0 &        0 &                  0 &                  0 &              0 &              0 \\
Tracking Eff.   &       0 &        0 &                  0 &                  0 &              0 &              0 \\
Trigger Eff.    &       0 &        0 &                  0 &                  0 &              0 &              0 \\
JES             &   0.005 &    0.006 &              0.017 &              0.019 &          0.004 &          0.003 \\
JER             &   0.006 &    0.002 &              0.006 &              0.003 &          0.001 &              0 \\
PDF             &   0.009 &        0 &              0.006 &              0.003 &          0.001 &              0 \\
top $p\_T$      &   0.011 &    0.012 &              0.016 &              0.017 &          0.001 &          0.001 \\
\hline
Total           &  0.0162 &   0.0136 &             0.0249 &             0.0258 &        0.00436 &        0.00316 \\
\hline
\end{tabular}
\caption{Systematic uncertainties of fit parameters from different sources. The total is the individual sources add in quadrature. }
\label{tab:sys-err}
\end{table}


\clearpage
\section{Results}
\label{sec:results}

The result of measuring $A_{FB}$ and $R_{q\bar{q}}$ from 19.6 $fb^{-1}$ of 8 TeV proton-proton collision data collected by CMS experiment in 2012 is given below. It is based on the binned likelihood fit of MC simulated templates ( and Data driven QCD multijets template for e+jets ) to 45321/42923 mu+jets/el+jets data events. 
\begin{itemize}
\item $ A_{FB} = 0.045  \pm 0.050 \, (stat) \pm 0.016 \, (sys)$

\item $ R_{q\bar{q}} = 0.120  \pm 0.006 \, (stat) \pm 0.014 \, (sys) $
\end{itemize}


The expected distribution of observed data events for the best fit is compared with actual observed data in the figures below. Fig.[\ref{fig:postfit combined}] shows the combined event distribution, by summing over $e+jets$ and $\mu+jets$ channels and over lepton charge types. The Fig[\ref{fig:postfit c*}] - Fig.[\ref{fig:postfit mtt}] shows the individual post fit comparisons for all four observable, which are fit simuteneusly as described in Section.[\ref{sec:lepton combination}]. The fit agrees with data reasonably well. 

In order to compare to the expected values of $A_{FB}$ and $R_{q\bar{q}}$ suggested by NLO simulation, we fit the pseudo data events which is the combination of MC simulated events. The same MC that is used for building the templates are taken to form the pseudo data. These MC samples are normalized to the same integrated luminosity of analyzed data. The fit central value and statistical uncertainty are also listed in the Table.\ref{tab:result_ref}.

We found that the measured $A_{FB}$ and $R_{q\bar{q}}$ are consistent with the expected values from NLO MC simulations. We note that just by simple counting definition of $A_{FB}$, we found $A_{FB}=-0.02$ using the generated $c*$ for $q\bar{q}\rightarrow t\bar t$ events from our signal MC samples. So we actually managed to extract this value with our template fit. On the other hand, this suggest our signal MC samples, which is generated using aMC@NLO may not be accurate enough to describe $A_{FB}$ from SM theory.

We also compared our measurement with the result of both D0 and CDF experiments of Tevatron, which are the measurement of $A_{FB}$ of $e/\mu$+jets channel combined based on full Tevatron Data of proton anti-proton collision at 1.96 TeV. Our results are consistent with the result of Tevatron \cite{d0,CDF2016}, and we get competitive uncertainty on $A_{FB}$ despite significant dilution from $gg$ initiated $t\bar{t}$ events. 

Finally we compared to the NNLO SM calculation \cite{Czakon:2014xsa}, which is consistent with our measurement as well. 

In conclusion, we measured the Forward-Backward Asymmetry of $q\bar q\rightarrow t\bar t$ process using the $l+4/5jets$ events from 8 TeV proton-proton collision in LHC, collected by CMS during 2012. We are able to measure $A_{FB}$ with good accuracy, and found the result to be consistent from NNLO QCD calculation as well as the latest results from D0 and CDF experiments in Tevatron. In addition, we managed to measure the fraction of $q\bar{q}$ initiated $t\bar{t}$ events , which is interesting in its own.



\begin{table}[hbt]
\begin{center}
\begin{tabular}{c|cccc}\hline
Parameter                 & Simulation   & Tevatron & SM Theory  \\
\hline
$A_{FB}$					  & $-0.018 \pm 0.052 $  & 0.106 $\pm$ 0.03 (D0), 0.164$\pm$0.047 (CDF)  &  0.095 $\pm$ 0.007\\
$R_{q\bar{q}}$			  & $ 0.132 \pm 0.015 $  & NA & ?? \\
\end{tabular}
\end{center}
\label{tab:result_ref}
\caption{Result of template fit to single muon data using 2012 8 TeV Data collected by CMS.  The expected value of parameters are from template fit to MC simulations. }
\end{table}


\begin{figure}[hbt]
  \begin{center}
    \includegraphics[width=0.49\linewidth]{postfit/lep_combo_x}
    \includegraphics[width=0.49\linewidth]{postfit/lep_combo_y}
    \includegraphics[width=0.49\linewidth]{postfit/lep_combo_z}
  \caption{\small Postfit plots of lepton and charge combined template after the fit (colored) and data (solid dots with error bar). All errors, including the shaded band in the Data/MC comparison plots, indicate Poisson error only.}
    \label{fig:postfit combined}
  \end{center}
\end{figure}

\begin{figure}[hbt]
  \begin{center}
    \includegraphics[width=0.49\linewidth]{postfit/el_f_minus_x}
    \includegraphics[width=0.49\linewidth]{postfit/el_f_plus_x}
    \includegraphics[width=0.49\linewidth]{postfit/mu_f_minus_x}
    \includegraphics[width=0.49\linewidth]{postfit/mu_f_plus_x}
  \caption{\small $c*$ projection of post-fit distribution of all four observable, as labeled in the figures. All errors, including the shaded band in the Data/MC comparison plots, indicate Poisson error only.}
    \label{fig:postfit c*}
  \end{center}
\end{figure}

\begin{figure}[hbt]
  \begin{center}
    \includegraphics[width=0.49\linewidth]{postfit/el_f_minus_y}
    \includegraphics[width=0.49\linewidth]{postfit/el_f_plus_y}
    \includegraphics[width=0.49\linewidth]{postfit/mu_f_minus_y}
    \includegraphics[width=0.49\linewidth]{postfit/mu_f_plus_y}
  \caption{\small $|x_F|$ projection of post-fit distribution of all four observable, as labeled in the figures.All errors, including the shaded band in the Data/MC comparison plots, indicate Poisson error only.}
    \label{fig:postfit xf}
  \end{center}
\end{figure}

\begin{figure}[hbt]
  \begin{center}
    \includegraphics[width=0.49\linewidth]{postfit/el_f_minus_z}
    \includegraphics[width=0.49\linewidth]{postfit/el_f_plus_z}
    \includegraphics[width=0.49\linewidth]{postfit/mu_f_minus_z}
    \includegraphics[width=0.49\linewidth]{postfit/mu_f_plus_z}
  \caption{\small $M_{t\bar t}$ projection of post-fit distribution of all four observable, as labeled in the figures.All errors, including the shaded band in the Data/MC comparison plots, indicate Poisson error only.}
    \label{fig:postfit mtt}
  \end{center}
\end{figure}