\chapter*{Introduction}

Particle physics studies the elementary particles, the fundamental building blocks of the universe, and their interactions. Since the discovery of electron in the end of nineteenth century, with the advancement of  experimental apparatus, more elementary particles has been discovered ever since. Theories has been developed along the way, from quantum mechanics to quantum field theory, and now all known elementary particles and forces have been described by a beautiful and unified model called Standard Model (SM) of particle physics. First proposed in 1960s, SM has been proven very successful with the ongoing experimental measurement, highlighted by the discovery of Higgs boson in Large Hadron Collider (LHC) in 2011. 

Before the discovery of Higgs boson, the discovery and the following study of another fundamental particle, top quark, is of critical significance. First discovered in 1995 in Tevatron collider at Fermilab, it is the heaviest elementary particle known to date, heavier than Higgs boson. Due to its large mass, top quark is often suspected to be different from all other quarks, and play a special role in Electro-Weak symmetry breaking. In addition, the large mass of top quark indicates a large Yukawa coupling to Higgs boson, so the top quarks also play an important role in Higgs boson production. For these reasons, and many others, it is of great interest to study many properties of top quark, and it is indeed the focus of this thesis to study one of the property of top quark pair production process. 

Since 2011, LHC has been running successfully, delivering huge amount of particle collision data at the highest collision energy ever registered in human history, first at 7 and 8 TeV in 2011 and 2012, in the so called LHC Run-1, then at 13 TeV from 2015 till now at LHC Run-2. A proton proton collider, LHC is a "top factory", producing much more $\ttbar$ events than Tevatron. Combining with the advanced detector like Compact Muon Solenoid (CMS), properties of top quarks such as mass, cross section of $\ttbar$ and single top, have been measured with highest accuracy to date. 


This thesis present a measurement of one of the properties of top quark, called Forward-backward Asymmetry ($\AFB$) of $\ttbar$ production. Pair production of top quark and anti-quark is the main source of top quarks in hadron colliders. There are two major production mechanism of $\ttbar$ pairs, the first is via gluon-gluon (gg process) fusion, the second is via initial quark and anti-quark annihilation ($\qqbar$ process). If a reference direction is chosen as the direction of initial quark, then the $\qqTT$ process is predicted not forward-backward symmetric in $\ttbar$ center of mass frame. Theoretical calculation using SM predict more top quark is produced in the forward hemisphere than the backward hemisphere. Equivalently, this effect can be observed as the excess of top quark over top anti-quark in the forward hemisphere, therefore it is also called Charge Asymmetry in many literatures as well. 

Forward-backward asymmetry of $\ttbar$ production draw the attention initially as both CDF and D0 experiments \cite{cdf,d0} reported observing a significant deviation from the SM prediction \cite{Kuhn:1998kw, Kuhn:2011ri, AguilarSaavedra:2012rx} in 2011. This motivated a lot research in model building that explain the anomaly with beyond standard model (BSM) physics. As the time of writing the thesis, with the improvement of theoretical calculation and updated measurements using full data set recorded in Tevatron \cite{Abazov:2014cca,CDF2016,tevatron_combine} , this anomaly has been reconciled. Regardless, as $\AFB$ only originated from higher order perturbative calculation using SM, it provides a precise test of SM. In addition, it is sensitive to the interference of SM process with a heavy BSM resonance that is hard to detect directly. For these reasons, it is still of great scientific interest to measure this effect in LHC.

Measuring the top quark forward-backward asymmetry at the LHC is considerably more challenging than at the Tevatron.  The $\ttbar$ cross section at the Tevatron is dominated by the $\qqbar$ process and the incident quark and anti-quark directions are reasonably well defined by the proton and antiproton beams. At the LHC, the production process is dominantly $gg$ and the the quark content of the initial state is symmetric. Since there can be no asymmetry from the gg initial state, these two effects significantly complicate the extraction of the asymmetry in $\qqTT$.  


Measurements \cite{ATLAS_measurement,CMS_measurement,Aad:2016ove,Sirunyan:2017lvd} done to date at LHC have focused on the determination of the so called charge asymmetry $A_C$ that is based upon the number of positively and negatively charged leptons observed in top pair events at large lepton rapidity. This quantity is diluted by the symmetric $gg$ initial states and uses only a fraction of the available information. 

All of the measurements done to date have been "empirical" in the sense that the measured quantity does not depend upon a model of the $\ttbar$ production mechanism although the interpretation of the measurements is model dependent. This thesis introduced a different approach. A simplified model for the production mechanism is adopted. This allows the use of a likelihood analysis to isolate the $\qqbar$ subprocess from the $gg$  and $qg$ subprocesses and from other backgrounds.  The adopted model is a leading order description of several possible BSM processes and is a reasonable approximation of the expected NLO QCD effects.

Using the template fit method proposed in this thesis, a measurement of inclusive $\AFB$ in $\ttbar$ production at LHC is performed and described in the thesis. The full 8 TeV data recorded by CMS is analyzed, and only the events in semileptonic decay of $\ttbar$ are used for the measurement. 
 


This thesis is organized as follows:

Chapter \ref{sec:theory} and \ref{sec:phenomenology} give an overview of the theoretical and phenomenological foundation of SM, cross sections and top quark physics.

Chapter \ref{sec:setup} introduces the experimental apparatus , including the LHC and the sub detectors of CMS. 

Chapter \ref{sec:modelling} described the signal and background process modeling and the data sample analyzed in this thesis. An overview of Monte-Carlo (MC) simulation procedure is given first, then the specific processes and the technical set up for the MC is listed. 

Chapter \ref{sec:objects} first described the physical objects reconstruction algorithms used in CMS and this thesis. Then, the signal event selection and the result of the selection is discussed.

Finally, Chapter \ref{sec:afb measurement} contains the details of the $\AFB$ measurements, which is the focus of this thesis and the original work of the author of this thesis as part of the team.