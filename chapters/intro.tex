\chapter*{Introduction}

Particle physics studies the elementary particles, the fundamental building blocks of the universe, and their interactions. Since the discovery of electron in the end of nineteenth century, with the advancement of experimental apparatus, more elementary particles have been discovered. Theories, from quantum mechanics to quantum field theory, have been developed along the way and now all known elementary particles and forces can be described by a beautiful and unified model called the Standard Model (SM) of particle physics. First proposed in 1960s, the SM has been proven very successful in describing ongoing experimental measurements, highlighted by the discovery of the predicted Higgs boson at the Large Hadron Collider (LHC) in 2011. 

Before the discovery of the Higgs boson, the discovery and the ensuing study of another fundamental particle, the top quark, was of critical significance. First discovered in 1995 in the Tevatron collider at Fermilab, it is the heaviest elementary particle known to date, heavier than the Higgs boson. Due to its large mass, the top quark is suspected to be different from all other quarks, and to play a special role in electroweak symmetry breaking. In addition, the large mass of the top quark indicates that it has a large Yukawa coupling to the Higgs boson, therefore top quarks also play an important role in Higgs boson production. For these reasons and many others, it is of great interest to study many properties of the top quark, and indeed the focus of this thesis is the study of one of the property of top quark pair production process. 

Since 2011, the LHC has been running successfully, delivering a huge amount of particle collision data at the highest collision energy ever registered in human history, first at 7 and 8 TeV in 2011 and 2012, in the so called LHC Run-1, then at 13 TeV from 2015 until now at LHC Run-2. A proton-proton collider, LHC is a "top factory", producing many more $\ttbar$ events than were produced by the Tevatron. Using advanced detectors like the Compact Muon Solenoid (CMS), the properties of the top quark such as its mass and production cross sections have been measured with the highest accuracies to date. 

This thesis presents a measurement of one of the properties of the top quark called the Forward-Backward Asymmetry ($\AFB$) of $\ttbar$ production. Pair production of top quarks and anti-quarks is the main source of top quarks in hadron colliders. There are two major production mechanisms for $\ttbar$ pairs, the first is via gluon-gluon (gg process) fusion and the second is via initial quark and anti-quark annihilation ($\qqbar$ process). If a reference direction is chosen as the direction of initial quark, then the $\qqTT$ process is predicted to be forward-backward asymmetric in the $\ttbar$ center of mass frame.  SM theoretical calculations predict that more top quarks are produced in the forward hemisphere than in the backward hemisphere. Equivalently, this effect can be observed as an excess of top quarks over top anti-quarks in the forward hemisphere and is therefore also called the Charge Asymmetry in many literatures as well. 

The $\ttbar$ Forward-Backward Asymmetry initially drew considerable attention in 2011 as the CDF and D0 experiments \cite{cdf,d0} at the Tevatron reported observing significant deviations from the SM prediction \cite{Kuhn:1998kw, Kuhn:2011ri, AguilarSaavedra:2012rx}. This motivated considerable theoretical work to explain the anomaly with beyond standard model (BSM) physics. At the time of the writing of this thesis, improvements in the theoretical calculations and updated measurements using the full data sets recorded at Tevatron \cite{Abazov:2014cca,CDF2016,tevatron_combine} have reconciled the anomaly. Regardless, as $\AFB$ only originated from higher order perturbative calculation using SM, it provides a precise test of SM. In addition, it is sensitive to the interference of SM process with a heavy BSM resonance that is hard to detect directly. For these reasons, it is still of great scientific interest to measure this effect in LHC.

Measuring the top quark forward-backward asymmetry at the LHC is considerably more challenging than at the Tevatron.  The $\ttbar$ cross section at the Tevatron is dominated by the $\qqbar$ process and the incident quark and anti-quark directions are reasonably well defined by the proton and antiproton beams. At the LHC, the production process is dominantly $gg$ and the the quark content of the initial state is symmetric. Since there can be no asymmetry from the gg initial state, these two effects significantly complicate the extraction of the asymmetry in $\qqTT$.  

Measurements \cite{ATLAS_measurement,CMS_measurement,Aad:2016ove} done to date have focused on the determination of the so called charge asymmetry $A_C$ that is based upon the number of positively and negatively charged leptons observed in top pair events at large lepton rapidity. This quantity is diluted by the symmetric $gg$ initial states and uses only a fraction of the available information. 

All of the measurements done to date have been "empirical" in the sense that the measured quantity does not depend upon a model of the $\ttbar$ production mechanism although the interpretation of the measurements is model dependent. This thesis introduces a different approach. A simplified model for the production mechanism is adopted. This allows the use of a likelihood analysis to isolate the $\qqbar$ subprocess from the $gg$  and $qg$ subprocesses and from other backgrounds.  The adopted model is a leading order description of several possible BSM processes and is a reasonable approximation of the expected NLO QCD effects.

Using the template fit method described in this thesis, a measurement of the inclusive $\AFB$ in $\ttbar$ production using semileptonic $\ttbar$ events produced at the LHC is performed and described in the thesis. The full 8 TeV data recorded by the CMS Experiment is analyzed. 
 

This thesis is organized as follows:

Chapter \ref{sec:theory} and \ref{sec:phenomenology} give an overview of the theoretical and phenomenological foundation of SM, cross sections and top quark physics.

Chapter \ref{sec:setup} introduces the experimental apparatus , including the LHC and the sub detectors of CMS. 

Chapter \ref{sec:modelling} describes the signal and background process modeling and the data sample analyzed in this thesis. An overview of Monte-Carlo (MC) simulation procedure is given first, then the specific processes and the technical set up for the MC is listed. 

Chapter \ref{sec:objects} first describes the physical objects reconstruction algorithms used in CMS and this thesis. Then, the signal event selection and the result of the selection is discussed.

Finally, Chapter \ref{sec:afb measurement} contains the details of the $\AFB$ measurements, which is the focus of this thesis and the original work of the author of this thesis as part of the team.