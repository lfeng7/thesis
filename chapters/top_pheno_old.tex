\chapter{Top physics Phenomenology at the LHC}
\label{sec:phenomenology}
\chaptermark{Top physics Phenomenology at the LHC}

In Standard Model of particle physics, top quark is an up type quark in the third generation of fermions. It has spin 1/2, electric charge $Q=2/3$ and form a weak isospin doublet with bottom quark. It is also charged under SU(3) group, being a color triplets with three colors. 

As a result, top quark is affected by all three forces of nature described in SM: electromagnetic force, weak force and strong force. It couples to the respective gauge bosons via the following vertices described in Fig.\ref{fig:ttg}-Fig.\ref{fig:ttgamma} .

\begin{figure}[hbt]
  \begin{center}
    \includegraphics[width=0.48\linewidth]{feynman_rules/ttg}
  \caption{\small Top quark coupling to gluon via strong interaction.  [cite 1709.10508] }
    \label{fig:ttg}
  \end{center}
\end{figure}

\begin{figure}[hbt]
  \begin{center}
    \includegraphics[width=0.48\linewidth]{feynman_rules/ttW}
  \caption{\small Top quark coupling to W boson and bottom quark via weak interaction.}
    \label{fig:ttW}
  \end{center}
\end{figure}

\begin{figure}[hbt]
  \begin{center}
    \includegraphics[width=0.48\linewidth]{feynman_rules/ttZ}
  \caption{\small Top quark coupling to Z boson (right), via weak interaction.}
    \label{fig:ttZ}
  \end{center}
\end{figure}


\begin{figure}[hbt]
  \begin{center}
    \includegraphics[width=0.48\linewidth]{feynman_rules/ttgamma}
  \caption{\small Top quark coupling to photon via electromagnetic interaction.}
    \label{fig:ttgamma}
  \end{center}
\end{figure}

\clearpage

Top quark is special among all known six quarks due to its large mass. First discovered by Tevatron collider of Fermilab [cite] in 1995, it is measured having a mass of 173 GeV, almost as heavy as tungsten atom. Aside from Higgs boson, it is the only fundamental particle that is heavier than W boson, which is about 80 GeV. Due to the large mass difference between top quark and W boson, the phase space of top decay is very large, causing top quark to decay before being able to form any bound state of hadrons. This provide the opportunity for a careful study of QCD as top quark can be treated as a quasi-free quark during the production and decay process.

Another reason for the importance of top quark physics is due to the large coupling of top quarks and newly discovered Higgs boson. As the yukawa coupling between fermions and Higgs bosons is proportional to the mass of fermion, top quark has the largest coupling to higgs boson. The production of Higgs in LHC usually involves $\ttH$ coupling, in both dominant production mechanism of gluon-gluon fusion process (Fig.?) or associate production of Higgs and $\ttbar$ (Fig.?). Therefore the study of top quark mass and its coupling to Higgs is critical in testing the validity of Higgs mechanism, which is responsible for origin of mass via spontaneous symmetry breaking. 

\begin{figure}[hbt]
	\begin{center}
		\includegraphics[width=0.48\linewidth]{feynman_rules/ggH}
		\includegraphics[width=0.48\linewidth]{feynman_rules/ttH}
		\caption{\small Major higgs production mechanism in LHC. The left figure is via gluon gluon fusion, while right is via $\ttbar$ fusion.  [cite 1709.10508] }
		\label{fig:ttg}
	\end{center}
\end{figure}


With the running of Large Hadron Collider, combining a higher center of mass energy for proton-proton collision and much higher luminosity than Tevatron, LHC is indeed a top factory , opening the door for more precise measurement of properties of top quarks.

Many properties of top quark has been carefully studied in LHC, including the mass of top quark, the cross section of top anti-top pair production, the spin correlation of top anti-top pair productions and more. A good summary of latest results of top property measurements can be found in the literature. [cite review of LHC top measurement papers] 

Of many properties of top quark, in this thesis we exclusively focus on one particular property of top pair production, namely "Forward-Backward Asymmetry" ($\AFB$). It is the spacial asymmetry of top quark pair production with respect to the direction of incoming initial quark, as shown in Fig.[?]. At parton level it is defined as 
\begin{equation}
\AFB = \frac{N_{\mathrm{t bar t}}(c*>0)-N_{\mathrm{t\bar t}}(c*<0)}{N_{\mathrm{t bar t}}(c*>0)+N_{\mathrm{t\bar t}}(c*<0)}
\end{equation}

\begin{figure}[hbt]
  \begin{center}
    \includegraphics[width=0.48\linewidth]{general_fig/AFB_kin}
  \caption{\small Top quark coupling to photon via electromagnetic interaction.}
    \label{fig:ttgamma}
  \end{center}
\end{figure}

where $c*\equiv \cos(\theta*)$ and $\theta*$ is the production angle of top quark in $t\bar t$ center of mass frame. This quantity is interesting because according to SM $\AFB$ is zero at LO of perturbative QCD calculation, and only to appear with NLO calculation. So a good measurement of $\AFB$ provide an precision test of SM. 

In this chapter we will first review the production of top quark pairs in Chapter \ref{sec:Production}-Chapter \ref{sec:production in LHC}. Then we review the decay of top quarks. Finally we review the status of forward-backward asymmetry of top quarks pair production, which is the main topic this thesis. 

\section{Top Quark Pair Production}
\label{sec:Production}

\subsection{Leading Order}

In hadron coliders, top anti-top pairs are mostly produced via strong interactions. In leading order of perturbative QCD (order of $\alpha_s^2$) there are two categories of production mechanism. The first one is via quark anti-quark annihilation , which we denote as $q\bar{q}$ initiated top pair production. ( we use $q\bar q$ process for simplicity sometime in the later chapters of this thesis) The Feynman diagram of this process is shown in Fig.?. The differential cross section is:
\begin{equation}
\frac{d\sigma}{dc_*}(q\bar q;M^2) = \frac{\pi\alpha_s^2}{9M^2}\beta\left[1+\beta^2c_*^2+\left(1-\beta^2\right)\right]
\label{eq:qq_xsec_LO}
\end{equation}

In the equation above, $M$ is the invariant mass of $t\bar t$ pair. Top quark velocity in $\TTbar$ center of mass (cm) frame is $\beta=\sqrt{1-4m_t^2/M^2}$.  $\theta^*$ is The production angle between the initial state quark direction and the top quark direction in the $t\bar t$ cm frame, and $c_*\equiv\cos{\theta^*}$. $\alpha_s\equiv g_s^2/4\pi$ is the strong interaction strength constant, which is about 0.12 at the scale of Z boson mass.

\begin{figure}[hbt]
  \begin{center}
    \includegraphics[width=0.48\linewidth]{feynman_rules/qq_tt_LO}
  \caption{\small Feynman diagram for leading order parton level $q\bar q \rightarrow t\bar t$ process via strong interaction. [cite 1709.10508]}
    \label{fig:qq_tt_LO}
  \end{center}
\end{figure}

The second process of top pair production is the $s,t,u$ channels of gluon-gluon initiated process (simply denoted as $gg$ process in this thesis), as described by Feynman diagram of Fig.? The parton level differential cross section of this process in LO reads:

\begin{equation}
\frac{d\sigma}{dc_*}(gg;M^2) = \frac{\pi\alpha_s^2}{48M^2}\beta\left[\frac{16}{1-\beta^2c_*^2}-9\right]\left\lbrace\frac{1+\beta^2c_*^2}{2}+(1-\beta^2)-\frac{(1-\beta^2)^2}{1-\beta^2c_*^2}\right\rbrace
\label{eq:gg_xsec_LO}
\end{equation}

The t and u channel dominated gg process gives a distribution that is more forward and backward peaking, i.e. more likely in the phase space with higher $c*$, compare with $q\bar q $ process. This feature is crucial to motivate our template fit based measurement that is described in Chapter.?, which relies on the discriminating these two production mechanism of top quark pairs.

Note that according to Eq.\ref{eq:qq_xsec_LO}, LO calculation of $\qqTT$ process does not introduce $\AFB$, as the differential cross section is even in $c*$. The same is true for LO calculation of $\ggTT$

\begin{figure}[hbt]
  \begin{center}
    \includegraphics[width=1\linewidth]{feynman_rules/gg_tt_LO}
  \caption{\small Feynman diagram for leading order parton level $gg \rightarrow t\bar t$ process via strong interaction. The s,t,u channels are shown in left,middle,right figures respectively.  }
    \label{fig:gg_tt_LO}
  \end{center}
\end{figure}

\clearpage

\subsection{Next to Leading Order}
\label{sec:LO production}
In top pair production higher order corrections in perturbative QCD calculation is important due to the sizable value of strong interaction coefficient $\alpha_s\sim 0.1$ at the energy scale of this process. As the energy scale of top pair production is set by the large mass of top quark (173 GeV) the perturbative QCD calculation is able to give accurate predictions. 

Currently, Next-to-Leading order calculation is regarded as the standard for event generation of top-quark production, and being implemented in several event generator that is widely used in the experimental study of LHC. The ones that are used in this thesis are Powheg-box [cite] and aMC@NLO [cite]. The impact of NLO calculation to the total cross section compare with LO calculation can be as large as 30\%.[cite] The correction also has sizable effect in the shape of many kinematic distributions of top quark pair productions. 

More importantly, NLO process is the first higher order of calculation that predicts a non-zero forward-backward asymmetry via SM. As a result, we will limit our discussion of higher order calculation of QCD in top quark pair production to the NLO processes that contributes $\AFB$.

At parton level, the dominant source of $\AFB$ is higher order process of $\qqTT$. The symmetry originates from the interference of virtual radiation of gluon (box diagram) in Fig.(c) and Born process (LO) of $\qqTT$ in Fig.(d). [cite Kuhn and Rodrigo 98']. In order to avoid the infrared divergences when the momenta of the virtually radiated gluon in Fig. (c) goes to zero, it has to be summed with the interference between initial state and final state real gluon emission of $\qqTT$, which is described by Fig. (a) and Fig (b) . The inclusive asymmetry of this source is positive ,between 6\% and 8\% in most of the kinematic regions that can be probed in Tevatron or LHC. [cite Kuhn 2011] 

\begin{figure}[hbt]
	\begin{center}
		\includegraphics[width=0.8\linewidth]{feynman_rules/qqTT_NLO}
		\caption{\small Feynman diagram of next-to-leading order $\ttbar$ production that contributes to $\AFB$. In the figure, q indicates light quark, Q represent heavy flavor quark, in our case, top quark. [cite Kuhn]}
		\label{fig:qqTT_NLO}
	\end{center}
\end{figure}

Another non-negligible source of $\AFB$ is from the interference of QCD induced $t\bar t$ production and electromagnetic (QED) induced $t\bar t$ production. The color singlet configuration of QCD box diagram, Fig.? , interferes with the s-channel $t\bar t$ production via photon, which is also color singlet.

Because the $\AFB$ originates from NLO QCD terms involves real gluon radiation, instead of study $t\bar t$ production only, this thesis actually study the $\AFB$ in $t\bar t +\mathrm{jet}$ production, where jet is referring to the hadronization products of either quark or gluon. As the consequence of additional jet allowed, another process that is predicted to produce non-zero $\AFB$ should be mentioned, that is the "flavor excitation" in the $\qgTT$. It originates from the interference terms of amplitudes in the quark-gluon scattering. Like radiative correction of $\qqTT$ the matrix element of this calculation is in the order of $\alpha_s^3$, and the Feynman diagrams are shown in Fig.?. The parton level $\AFB$ originate from $\qgTT$ is much smaller than that from $\qqTT$ process.

\begin{figure}[hbt]
	\begin{center}
		\includegraphics[width=0.8\linewidth]{feynman_rules/qg_NLO}
		\caption{\small Feynman diagram of next-to-leading order $\ttbar+q$ production via quark gluon initial states, that contributes to $\AFB$. In the figure, q indicates light quark, Q represent heavy flavor quark, in our case, top quark. [cite Kuhn]}
		\label{fig:gg_tt_LO}
	\end{center}
\end{figure}

In the energy scale studied in this thesis, where the partonic center of mass energy $\sqrtS$ is below 1 TeV, the contribution of partonic level $\AFB$ with calculation at order of $\alpha_s^3$ originated from QCD $\qqbar$ process, the interference of QCD-QED, and from $\qgTT$ process are about 7\%,1\% and 0.1\% (??). In addition, the differential asymmetry defined in Eq.? is approximately a linear function of $\cstar$ for both $\qqbar$ and QCD-QED interference terms, while it is approximately quadratic for $\qgTT$ terms. For both reasons, in this thesis we attempt to measure the $\AFB$ originate from $\qqTT$ process from Data, while model the $\AFB$ originates from $\qgTT$ via MC simulation.

There is one subtlety when we use NLO MC simulated events to estimate the $\AFB$ at $NLO$ QCD level. In \cite{Kuhn&Rodrigus} the symmetric part of the cross section is using the result of LO ( $\alpha_s^2$ ) , while the direct counting result from NLO MC use NLO cross section for the symmetric part of the $\qqTT$ process.  Given that NLO correction increase the total cross section of this process by up to 30\%, this means the $\AFB$ derived by direct counting from NLO MC is only 0.7 times the value from theoretical prediction given in \cite{Kuhn&Rodrigus}               

\subsection{Top Quark Production in Hadron Colliders}
\label{sec:production in LHC}

In previous sections we described the parton level production of top anti-top pairs according to SM. In reality, quarks are never observed as isolated free particles. Because of strong interaction, it is the colorless bound states of quarks, mesons such as Pions (two quarks) and Bayons(three quarks) that is actually observed in both initial and final states of any scattering experiments. Because of the coupling constant of strong interaction is very large in low energy, it is not calculable using perturbation in this energy regime. However, because of the asymptotic freedom property of QCD, it can be shown that the phase space and kinematic distributions of scattering involve strong interactions can be described by the hard process in which large momentum transfer happens. An intuitive understanding is during the hadronization any process with large momentum transfer is suppressed as it corresponds to small coupling constant, so the hadronization products are collinear with the original quasi-free particles involves in the hard scattering process. 

Another result of asymptotic freedom of QCD is factorization theorem, which says the cross sections of scattering of hadrons can be calculated by convolute the parton distribution function (PDF), which describes the distributions of momentum fractions carry by partons that form the scattered hadron,  with the cross section of hard scattering process. We some times call the hard scattering process the parton level process. So the total cross sections of $\ttbar$ production in proton-proton or proton anti-proton collider can be represent in the following way:

\begin{equation}
\sigma_{p_1p_2\rightarrow \ttbar} = \sum_{(i,j)\in(q,\bar q,g)}\int_{0}^{1}\int_{0}^{1}(\sigma_{ij\rightarrow\ttbar})D_i^{p_1}(x_1,Q)D_j^{p_2}(x_2,Q)dx_1dx_2  
\end{equation}  

where $p_1,p_2$ can be either proton or anti-proton depend on the type of collider. $D_i^{p_1}(x,Q)$ is the PDF, which gives the probability of a parton , such as an up quark, shares the $x$ fraction of momentum of aits parent hadron of type $p_1$ (proton or anti-proton). Note that PDF also depends on the energy scale Q of the hard process. Similarly, differential cross sections in hadron colision can also be factorized by th convolution of PDF and parton level differential cross sections.

PDF can only be determined experimentally by fitting the data of hadron scattering or deep inelastic scattering experiments. One example of proton PDF at the energy scale of $\ttbar$ production is shown in Fig.? below. It shows that valence quark in proton (consist of two up quarks and one down quarks), the up and down quarks, tend to carry higher fraction of proton momentum than the fraction carried by sea quarks, such as $\bar u$ and $\bar d$ quarks, and gluons.  

Parton distribution function has major consequence for the production mechanism of top quarks in LHC vs Tevatron. The kinematic constraint of $\ttbar$ production requires the center-of-mass energy of initial partons ( which is also the invariant mass of produced $\ttbar$ pairs) $\sqrtS=\mtt=\sqrt(x_1x_2s)$ ($s$ is the center-of-mass energy of colision), to be at least two times the mass of top quarks, which is about 345 GeV. Such a large energy requirement indicate that both partons need to carry sufficient fraction of energy of the colided hadron in order to produce $\ttbar$ pairs.  

In Tevatron which is a proton anti-proton colision, the dominant parton level production mechanism of $\ttbar$ is via $\qqbar$ initial states (about 90\%). This is because both initial quark and anti-quark could be valence quark from proton or anti-proton in the colision, thus are likely to carry sufficient momentum for $\ttbar$ production. In contrast, the majority of $\ttbar$ pairs in LHC is produced via gg fusion process, as LHC is a proton-proton collider and it is unlikely for a anti-quark to carry sufficient momentum. Therefore $\qqTT$ is only about 10\% in LHC at 8 TeV. 

The consequence is that the $\ttbar$ $\AFB$ in LHC is much smaller than that in Tevatron due to the dilution of forward-backward symmetric $\ggTT$ process. Another challenge is that in Tevatron because of PDF, initial quark is almost always along the same direction of initial proton, so we can choose the direction of initial proton as the positive direction when we determine the production angle $\cstar$. In contrast, in LHC the initial beam set up is forward-backward symmetric. This makes the inference of quark direction at parton level difficult. The solution for this problem is by observing that the quark in the $\qqTT$ process likely carries more momentum than anti-quark. So the direction of direction of $\qqbar$ c.m system is usually the initial quark direction, especially in the case where the difference of momentum fraction of initial quarks, $|x_f| = |x_1-x_2|$ is larger, i.e. the $t\bar t$ c.m. system has higher boost in longitudinal direction.  

\section{Top Decay}
\label{sec:Decay}
 
 Top quark decays via weak interaction, almost exclusively to a W boson and a bottom quark, due to the form of CKM matrix, where $|v_{tb}|~1$. In addition, because of the mass of top quark 173 GeV is much higher than the mass of W boson, the top quark decay width is so large that the top quark decays happens in a shorter time than the hadronization time. So top quark decay can be calculated with perturbative QCD very accurately.
 
 The final states of $\ttbar$ events can be categorized based on the decay mode of two W bosons. W bosons can either decay to a quark anti-quark pair (u,d,c,s type),which is called hadronic decay, or a lepton-neutrino pair (leptonic decay),  both via weak interaction. So there are three different decay modes of $\ttbar$ pairs:
 \begin{itemize}
 	\item All-hadronic: Both W bosons decay hadronically, $\ttbar\rightarrow b \bar b jjjj$
 	\subitem The most abundant decay mode, which is about 44\% of all $t\bar t$ events, as shown in Fig.\ref{fig:TT_composition} . It is not as clean as the other two channels, with large W+jets or QCD multijet backgrounds.
 	\item Semileptonic: One of the W bosons decay leptonically, another decays hadronically , $\ttbar \rightarrow b\bar b l\nu j j$, and the Feynman diagram is shown in Fig.\ref{fig:TT_semileptonic}
 	\subitem We only consider the case where lepton is electron or muon, and ignore the case where lepton is tau, because tau will quickly decay which is complicated to reconstruct. 
 	\subitem This channel is optimal for studying $\AFB$ for several reasons: It is relatively clean due to the requirement of a electron or muon and two bottom quarks; it is relatively abundant, about 30\% of all $\ttbar$ events, providing sufficient statistics for the measurement; It has only one neutrino, making it relatively easy to correctly reconstruct the momentum of top and anti-top from their decay products.  
 	\item Dileptonic: Both W bosons decay leptonically, $\ttbar \rightarrow b \bar b ll\nu\nu$
 	\subitem This is the cleanest channel, due to the requirement of two leptons. The problem is the relative small abundance ( about 4\%), and existence of two neutrinos makes reconstruction of top anti-top momentum challenging, thus not suitable for our purpose.  
 \end{itemize}  

As mentioned in Section.\ref{sec:LO production}, we allows an extra hard gluon from ISR or FSR radiation process associated with $\ttbar$ production. As a result, the final states we studied in this thesis is $l+4/5jets+MET$, where MET means missing transverse energy which corresponds to the transvers momentum of neutrino. 

\begin{figure}[hbt]
	\begin{center}
		\includegraphics[width=0.6\linewidth]{general_fig/top_pair_branching_frac}
		\caption{\small The branching fractions of all channels of $\ttbar$ productions.}
		\label{fig:TT_composition}
	\end{center}
\end{figure}

\begin{figure}[hbt]
	\begin{center}
		\includegraphics[width=0.6\linewidth]{general_fig/feynman_ttbar_ljets_beamline}
		\caption{\small The Feynman diagram of semileptonic decay of $\ttbar$ pairs. There is another decay process with $W^-\rightarrow l^-\bar \nu$ not shown here.}
		\label{fig:TT_semileptonic}
	\end{center}
\end{figure}

 
\section{Forward Backward Asymmetry}
\label{sec:AFB}

In this section I will give an overview of the status of current AFB and $A_C$ measurement in Tevatron and LHC.